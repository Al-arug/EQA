\documentclass[11pt, a4paper]{article}

\usepackage{mlt-thesis-2015}

% With Xetex/Luatex this shouldn't be used
%\usepackage[utf8]{inputenc}

\usepackage[english]{babel}
\usepackage{graphicx}
\usepackage{setspace}
\usepackage{float}
\usepackage{hyperref}
\usepackage{subfig}



\title{Embodied question answering in\\ robotic environment}
\subtitle{Automatic generation of a synthetic question-answer data-set }
% NI 21-09-27: author line
\author{Ali Aruqi}

% NI 21-09-27: I think we need a more specific subtitle, at the moment this describes what the original paper has done (e.g. created a synthetic datasets of questions and answers), but in current study you have extended their dataset and generated a different type of questions. Also, you have discovered and analysed shortcomings of the EQA dataset. 'Embodied Question Answering in a 3D World: Towards More Realistic and Diverse Questions and Answers'? Just an idea.

\begin{document}

%% ============================================================================
%% Title page
%% ============================================================================
\begin{titlepage}

\maketitle

\vfill

% NI 21-09-27: \iffalse (in the beginning) and \fi (in the end) can be used to comment out big chunks of text; more convenient if you don't want to spend time on placing % signs for each line
\iffalse
\begingroup
\renewcommand*{\arraystretch}{1.2}
\begin{tabular}{l@{\hskip 20mm}l}
\hline
Master's Thesis: & 30 credits \\
Programme: & Master’s Programme in Language Technology\\
Level: & Advanced level \\
Semester and year: & Spring, 2020\\
Supervisor & Simon Dobnik, Nikolai llinykh\\
Examiner & Staffan Larsson\\
Report number & number will be provided by the administrators \\
Keywords & Artificial Intelligence,  Natural Language Processing,\\ Computer Vision, \ \ Visual Question Answering, Synthetic Data-sets
\end{tabular}
\endgroup
\fi

% NI 21-09-27: There was some weird formatting: text in the second column were always outside of the page itself. I have re-formatted things. Very similar to what was before, but a bit different. Use tabular* to fit text based on the linewidth. Use p{} in tabular environment to fit text in the table instead of l/r/c.
\begingroup
\renewcommand*{\arraystretch}{1.2}
\begin{tabular*}{\textwidth}{l@{\hskip 20mm}p{0.6\linewidth}l}
\hline
Master's Thesis: & 30 credits\\
Programme: & Master’s Programme in Language Technology\\
Level: & Advanced Level \\
Semester and year: & Autumn, 2021\\
Supervisor & Simon Dobnik, Nikolai llinykh\\
Examiner & Staffan Larsson\\
Report number & (number will be provided by the administrators)\\
% NI 21-09-27: Star the keyword list with more specific phrases, which apply to your thesis. Then, gradually make it more general. Not sure how many keywords are required, but 4-5 is a good number, in general.
Keywords & Embodied Question Answering, Question Generation, Spatial Relations, Synthetic Datasets, Multi-Modality
\end{tabular*}
\endgroup


\thispagestyle{empty}
\end{titlepage}

%% ============================================================================
%% Abstract
%% ============================================================================
\newpage
\singlespacing
\section*{Abstract}
% NI 21-09-27: copying some of my comments from the previous review, I think they are still valid.
% What about structure like that: start with the general description of the task, specify what is the problem, describe your proposal to solve this problem, conclude what are your results and how your solution has helped to improve the state of this dataset.
% (draft, feel free to edit) Embodied Question Answering is the task of answering questions about virtual environment. In this task, the agent is spawned in a 3D world, where it is required to navigate to the object that the question asks about. The agent is trained on simple questions which do not require any complex spatial reasoning, typical for navigation in a real world (e.g., colour questions). Therefore, one problem with this dataset is the lack of questions which require complex spatial grounding  to be answered. In this work, we extend the original EQA dataset with more complex questions, which ask about size and position of the objects. Specifically, we…(details about questions). Our results show that…(concrete results). This work provides initial investigation of...(describe why this work is important and where it can be applied/how it can be extended).

Our work extends a dataset for Embodied-Question-Answering. Embodied question answering is the task of asking a robot a question about objects in a 3D environment, where the agent is expected to navigate the environment and find the entities in question and answer. The answer system consists of navigation and VQA components.  Each question in the dataset is an executable function that could be run in the environment to yield an answer.  The published dataset for EQA is EQA-V1, and it is a limited dataset that includes only two types of questions, color and location questions. We use the navigational data, required for training the system, from EQA-V1 and generate new questions of two more types, size and spatial questions. Our data extension is intended to better train the system and enhance its ability in performing the task. 

\thispagestyle{empty}

%% ============================================================================
%% Preface
%% ============================================================================
\newpage
\section*{Preface}

Acknowledgements, etc.

\thispagestyle{empty}

%% ============================================================================
%% Contents
%% ============================================================================
\newpage

\begin{spacing}{0.0}
\tableofcontents
\end{spacing}

\thispagestyle{empty}

%% ============================================================================
%% Introduction
%% ============================================================================
\newpage
\setcounter{page}{1}

% NI 21-09-27: General advice: write every sentence in the new line. This would significantly simplify tracking and editing for all three of us.

% NI 21-09-27: When citing, you do not have to wrap citations in (). This is done automatically. Also, there are different ways to cite when you simply mention the reference to support something that you are saying, or when you are describing what has been done by specific reference. Use \citep{} for the first case, \citet{} for the second case. I have formatted the first paragraph as an example.
% NI 21-09-27: Sometimes, a different style would work as well - \citep{} when mentioning references and \cite{} when embeddings them in text (instead of \citet{}).

\section{Introduction}
\label{sec:intro}

An intelligent robot must be able to understand and resolve references in its environment \citep{russell1995artificial}.
Our human ability to interact with our visual surroundings, manifested in language, stems from faculties such as perception and memory \citep{regier1996human}.
Perception, in particular, is central to our physical experience of the world \citep{barsalou1999perceptual}.
We conceive the physical world through perception, and we express our %conceptualization - NI 21-09-27: use British spelling consistently
conceptualisation of the perceptual experience in words \citep{lakoff2008metaphors}.
% NI 21-09-27: I use the next sentence as an example of using \citet{}. You can see that the reference now looks like it is a part of the text. You would normally use \citet{} when you want to embed your reference in the text or maybe when you want to give details about someone's work.
%Therefore, the exhibition of intelligent %behavior
%behaviour is necessitated by having a notion of meaning that associates 'words' with the visual/physical world \citep{nilsson2007physical}.
Therefore, as \citet{nilsson2007physical} argues, the exhibition of intelligent behaviour is necessitated by having a notion of meaning that associates 'words' with the visual/physical world.

\subsection{meaning}

% NI 21-09-27: Write ``chair'', not ''chair''. Use ``[word]'', not `[word]' or '[word]' or any other variation, stick to ``[word]''. This is sort of general rule that everyone follows.

% NI 21-09-27: You often want to format things differently if they are used as something OTHER than plain text. For example, you should reformat (c, h, a, i, r), this can be done in many ways - the main thing is that your formatting is consistent throughout the whole manuscript. For example:  {[ \texttt{c}, \texttt{h}, \texttt{a}, \texttt{i}, \texttt{r} ]}. If you put this in text, you will see that this looks a bit different. There is no right format for each case, people normally use something they have seen in other papers. Again, the main point: be consistent and use the same formatting every time you talk about letters in the word (as in this case).

The meaning of words is not a mere psychological phenomenon. Concrete nouns, for example, have references in the physical world, with physical properties indicated by their meaning. The meaning of a word is, thus, not only bound up with linguistic characters and mental notions but also with some physical representation in the world. For example, the word ``chair'' is represented by its token-characters (c, h, a, i, r), contains a perceptual symbolism(mental understanding of the chair's attributes and functions), and refers to an entity with physical features in the world(\cite{mooney}).

% NI 21-09-27: a good place to refer to the paper by Bender and Koller: https://aclanthology.org/2020.acl-main.463/ and all other research that shows that meaning of words is grounded in reality. You can emphasise the need for grounded agents, which is a hot topic these days (e.g., big language models do not capture real meaning of words, because they are not grounded). This need also strengthens and motivates your current work - you are working on the multi-modal system that answers questions (some sort of interaction setting, not just static captioning).

In this triangular definition of meaning, 'vision' has an integral part of the meaning in which it represents the physical world to language. To recognize a chair, one should, for example, identify the existence of legs, seats, their sizes, and geometric shape in a visual scene. % NI 21-09-27: I see what you want to say, but we also often do not need to see object as a whole. I can see legs of the table to predict that the whole object is the table. This happens because I conceptually know what are the parts of a prototypical table since I have encountered tables before and learned how they look. Visual recognition does not form only from perception, our experiences and knowledge also play a role. This is related to my next point about rotten food: if I know that bad fruit colour indicates that the food is rotten (because I have seen and tasted it before in this state), I can already predict that the food is rotten by merely looking at it.
These properties of the physical reference of a chair can be clearly represented in a visual form. Therefore visual recognition is part of the conceptualization process that forms the perceptual symbol of an entity.(\cite{barsalou1999perceptual})

Perceptual information, however, is more than just visual information. The properties of an object include other sensory information such as the smell, taste, and texture of an object. For example, the meaning of rotten food could be more understood if the food is tasted.
% NI 21-09-27: not necessarily, the food might also visually look rotten, this would be enough to infer that it is not fresh.

The formation of a symbolic representation (meaning) requires more than the recognition of perceptual information. The construct of a meaning (symbol) could only be formulated by the existence of knowledge about the relations of the attributes that form an entity.(\cite{barsalou1999perceptual}) refers to the process of forming a symbol as 'componential' or schematic, meaning that a  notion of meaning is a scheme that is conceptualized or constructed. These schemes can either be logically constructed in our mind in terms of holding valid truthiness about the world, or they can be based on incoherent conceptualization forming a false knowledge. 

The full meaning is the perceptual representation and the knowledge about it. The meaning of rotten apple is fully understood when we construct knowledge of the negative aspects of eating it. For example, rotten apples have red-brown colors, and the stomachache that results by eating them leads us to form the belief that red-brown apples are different from all-red apples, not only by color but also other health/taste attributes, so we classify red-brown apples in a different category called "rotten apples." The knowledge about health implications and the attributes such as the colors and smell of a rotten apple help us categorizing the rotten apple in the category of rotten food. (\cite{lakoff2008metaphors}) explains that this attributive characterization can be expressed, for example, in the way we do prototyping and categorization of entities.

% NI 21-09-27: the story above is scattered a bit. Start with saying which conditions are required to capture meaning. Then, go into details of each of these conditions. You can go from saying that meaning is not merely in texts, but also around us. We use words to refer to the world around and communicate it. Frege's example of Morning Start/Evening Star - two different phrases are used to refer to the same object. Then, you would move to perception (vision) and knowledge (experiences, etc.). As I can see, you want to show that meaning is not only in texts, but also in vision, taste, sensor information. Try to go from text to multi-modality here.


\begin{figure}[H]
\centering
\includegraphics[scale=0.5]{images/symbolsyststem.png}
\caption{\cite{roy2005semiotic}}
\label{fig:roy}
\end{figure}.
% NI 21-09-27: every figure should be accompanied by a caption that describes what this figure shows. Citation or simple phrase is not enough.

% NI 21-09-27: below, you are moving to meaning in interaction with the world, right? A smoother connection with previous text is needed.
% NI 21-09-27: when referring to figures, write Figure~\ref{fig:roy}

Interactions in the semantic world have an exchangeable nature. The interaction allows us to form meaning, and the formed meaning shapes our language and actions. In Figure~\ref{fig:roy} we see that the outcomes of our interaction in the world includes not only linguistic implications but also affects the actions(\cite{roy2005semiotic}). 'schemes about the world' are the beliefs we make from the interactions. For example, our negative experience with the red-brown apple made us form the  "belief" that rotten apples are bad. The knowledge that "rotten apples are bad" influences our future actions- makes us not eat the apples with attributes of "rotten." 

The process of comprehending meaning through associating attributes with each other to form a belief or draw a conclusion denotes the notion of reasoning. The process of classifying the apple as rotten includes multiple abstractions. We might first identity the apple by its general shape structure, then recognize, from previous experiences, that apples in red-brown color are not like all-red apples, then conclude that the apple is rotten. Reasoning is the ability to take the logical steps to conclude.  
 


\subsection{Grounding meaning}

The approaches to ground meaning(form meaning) vary depending on the aspect of meaning that each approach focuses on. The different methods we review below-approaching meaning as a mental notion, a map of connected knowledge nodes,  vision and language representation, or the combination of different aspects of meaning representation. 

Word-meaning in a Vector Semantic Space (VSM) represents a mental aspect of meaning notion(\cite{Turney_2010})). Space can be understood by imagining our minds as a space that we allocate meaning representation in them. In VSM, the mind (represented as neural language model) is an artificial space where word meanings are allocated at different distances from each other depending on their categorization, such as a rotten apple is closer to fresh apples than to a chair. 

There are multiple hypotheses to representing word-meaning in a Vector Semantic Space(VSM). Distributional hypothesis is a popular example of word representation in VSM. The premise of this approach is that language is compositional, and word-meaning can be defined by its context--The meaning of a word is represented by the word and the words surrounding it \cite{Turney_2010}. The formulated word meaning representation is known as "word-embeddings" \cite{mikolov2013distributed}. Using language to define language has proven promising in inferential tasks such as inferring that "university" and "student" are close to each other given their common context of "education." 

% NI 21-09-27: I am not exactly sure if you really need the paragraphs above. In the context of this thesis, grounded meaning is about multi-modal meaning. Word embeddings are not multi-modal and vector semantic space does not really fit here. This is about general meaning representations for computation, while you are focusing on several modalities for meaning. Maybe you can re-phrase the stuff above and just say that meaning can be uni-modal (mention word embeddings etc)?
 
% NI 21-09-27: The focus in this subsection should be on multi-modal grounded meaning. You need to add more discussion here. https://aclanthology.org/P12-1015.pdf - this paper has some good references in the introduction about multi-modal meaning. Another good reference: https://aclanthology.org/N10-1011.pdf.
Representing meaning with a combined linguistic and visual representation is the second example of an extensively researched field. Early research in combining vision and language used probabilistic learning by aiming at drawing an alignment between sentences, phrases, and words with the corresponding perceptual representations.\cite{790410}. An approach to probabilistic learning estimates the probability of a grammatical entity(text) being related to a perceptual representation \cite{6751319}. A second probabilistic method is classifying each word in a sentence through the probability distribution of words over a perceptual representation. In \cite{matuszek2012joint} \cite{larsson2015formal} we see examples of connecting entities of formal semantics(First Order Logic) with perception. 

There are examples of research that attempt to incorporate knowledge graphs in the representation of meaning. Examples, \cite{zhu2015building}, \cite{zhu2014reasoning}. 
%Despite the variety of methods in forming meaning, the comprehension of a full meaning cannot be formed in exclusion of any of its aspects. Even abstract concepts can sometimes have a physical/visual representation such as Art. Art works cannot be understood if we do not perceptually perceive them and connect them to some mental notion of knowledge. In order to achieve such high cognition abilities in robots, scientist and researchers are continuously introducing finding new methods and approaches to improve the computer's abilities.  

%In  robots, a full comprehension of meaning is essential for its usability. 
%the ability to understand the visual aspect of meaning is essential for the robot's ability to interact and preform action. 

%A paragraph about neural language models for vision and language (intro for image captioning)





%\cite{dunning1993} introduced a well-known method for extracting
%collocations. Bilingual data can be used to train part-of-speech
%taggers \citep{das2011}. Another one: \citep{cortes2014}

%Testing Unicode: Göteborgs universitet

%\textit{Testing} \textbf{testing} \textsc{testing} some font series.

%Testing a formula:
%\[
%P(X) = \sum_{i=1}^N P(A_i) P(X|A_i)
%\]

%Testing a table:
%\begin{table}[htbp]
%\begin{center}
%\begin{tabular}{c|c}
%cell 1 & cell 2 \\
%\hline
%cell 3 & cell 4
%\end{tabular}
%\caption{This is a table.}
%\end{center}
%\end{table}



\newpage

\section{background}
\label{sec:background}






% NI 21-09-27: subsection title - Multi-Modal Tasks
\subsection{image captioning}

% NI 21-09-27: https://arxiv.org/abs/1601.03896 - image captioning reference, you need to make a smoother introduction.
Image captioning is an extensively researched field where visual-grounding is at the  center of its focus. The methods of image captioning provide insights helpful for tasks that combine vision an language. In this section we review two main methods used in image captioning,feature extraction and attention mechanisms.   


\subsubsection{Feature extraction}

% NI 21-09-27: If you are not talking about the first group and only about the second group, do not talk about this distinction. You need to say that image captioning models typically use encoder-decoder architecture, where Encoder is a pre-trained CNN and Decoder is a recurrent network, e.g. LSTM.
Feature extraction methods can be divided into two main groups. The first group relies on statistical language models. The second group relies on encoder-decoder neural network model that deep extracts features.\cite{Imagecap}

\paragraph{Encoder-Decoder(CNN-RNN)}
% NI 21-09-27: A figure of encoder-decoder captioning model would be really great here. It would simply reading of the text.
% NI 21-09-27: when you use \paragraph{}, you typically would not separate texts later with new lines. It looks like you have a lot of smaller paragraphs in a big paragraph section. This is a bit unnatural, so try to compress text and add figures to it. This would make it more visually pleasing. I think that the main comment here is that a lot of text here can be compressed in shorter, more condensed version of it. Use figures and visualisations to support your text instead of writing all details extensively.

Convulutional neural networks are at the core of feature extraction methods. CNN applications, nonetheless, take a vital role in many computer vision tasks. We see CNN and its modified models (such as recurrent-CNN) used in tasks as object recognition \cite{liang2015recurrent} \cite{objdet} \cite{Ren2015FasterRT} , image classification \cite{simonyan2014very} \cite{imclassfication},and  semantic segmentation \cite{hariharan2015hypercolumns} \cite{imseg}. 

% NI 21-09-27: The way how CNNs function should be accompanied by the figure (comment above), and then you can follow this figure in text and describe what they take as an input and how they process it. You cite Vinyals who has excellent description in the paper, that you can look at and do something similar.
% NI 21-09-27: There are many ways to represent image. You can do it as a single vector for the whole image (Vinyals). Or you can split it into either a uniform grid of cells or detect objects - briefly mention Anderson et al. 2017 as a better approach to represent images https://arxiv.org/abs/1707.07998. The latter method is state-of-the-art in captioning.
A main reason for using CNN for image processing  is its ability to reduce the high dimensionality of images. Image features contain large sizes represented in pixels which would require large number of parameters to train. CNN reduces the dimensions of an image by learning how to process a matrix from a large window such as 250x250 pixels into a smaller one as 25x25. Through computing the convolution values of the image matrices and executing pooling computations, this process reduces the image into a smaller representation. The latter reduces the computational load and helps in processing and classifying the images faster.

% NI 21-09-27: add show, attend and tell paper here. And describe attention mechanism - it is an important asset for captioning and also for your embodied systems. You might want to refer to discussion on attention later in the thesis, so it would be nice if you have a paragraph about it.
The encoder-decoder caption generation has a CNN encoder and an RNN decoder.\cite{vinyals2015tell} is an example of an end-to-end neural captioon generation model. In the neural model the CNN process the image features, and the last hidden layer passed to an RNN to generate a description. This method is a sequence modeling that is similar to machine translation. This means that image features are translated into words. The sequence is predicted by finding the probability of a certain description from a corpora given the features of an image. 

RNN are known to be used widely in language technology applications. Rnn is used , for example in  text-to-speech \cite{arik2017deep} and  machine-translation \cite{cho2014learning},\cite{Wu2016GooglesNM}. The advantage that the RNN gives to these tasks is that the output size is not fixed and that each output depends on the previous one. Such an incremental-sequence prediction is suitable for sentence predictions in respect to word dependency.  

% NI 21-09-27: I am not sure about the next two paragraphs. I think that generally saying that LSTMs are used instead of RNNs because of the vanishing gradient problem (and giving one reference) would be enough. It is not neecessary to go into RNN vs LSTM because you are not addressing it in any form in the experiments.
RNNs have a issue of vanishing gradient-descent. The gradient descent is an optimization algorithm that minimizes the error calculated in the loss function. Optimization, in brief description, is important for the learning process. It updates the model's parameters which determines the  direction taken in the next time-step. This information is calculated given the input-output and the values of the parameters from the previous time-stamps. The gradients is reduced at every step due the value deductions in the activation function. When the gradient is reduced to almost zero value, it will be updating the parameters with no useful values, and therefore, learning seizes to improve. 

Long-Short-Memory network (LSTM)  provides a good alternative for avoiding the disappearing gradient. The gradient in RNNS vanishes in long sequences where the gradient keeps reducing. The architecture of the LSTM allows it to keep information stored for very long sequences. The latter gives it the ability to control the values of the gradient by updating it with information stored in the 'forget gate' form previous steps, preventing the gradient from vanishing.  


% NI 21-09-27: I would also remove this paragraph, this provides some information which is not that much necessary for general overview of captioning models.
\paragraph{Feature extraction- statistical language model}

Statistical language model, as in \cite{fang2015captions}, generate descriptions in three stages. First it detects words in an image using a convolutional neural network (CNN) for extracting image features. The incorporation of language at this stage happens using multi-instance learning(MIT)\cite{zhang2005multiple}. The second stage, the statistical language model detects the most likely sentence to make of the words from a pre-defined corpus. In the third stage the sentences undergoes a re-ranking stage where the sentences are combined to generate captions. 


%\subsubsection{Attention-mechanisms}


% NI 21-09-27: overall structure of the whole subsection above should be something like that:
% 1. what is the task of image captioning? What is it used for?
% 2. describe encoder-decoder architecture with figure and texts (describe how encoder takes image and how it gives information to the decoder, which generates caption
% 2.1 describe what is the input to the CNN and what is the input to LSTM in a bit more detail
% 2.2 talk about attention (show attend and tell paper) and say what are the benefits of attention (attention is better because the model can look at the most salient regions/parts in the image, the system becomes more focused on semantically important parts of the image, which is also similar to how humans describe images - we do not focus on every single detail, instead we build concepts from what we see and mention only the most important bits of what we see.
% 2. It would be nice to say why we are talking about encoder-decoder captioning models here. Because these architectures are adopted in more complex multi-modal tasks as part of the system, e.g. embodied question answering.


\subsection{Dialogue and VQA}

% NI 21-09-27: restructure the section and start with VQA, then move to visual dialogue. Embed the following reference in text: https://arxiv.org/abs/1612.00837. The paper shows issues with VQA dataset, e.g. bias towards language, and introduces a second version of the vqa dataset, more balanced. This paper can be described in a way to emphasise your own research question, because you are doing very similar thing.

In this section of the text we discuss the capabilities of computers to exhibit more intelligent behaviour. Image-captioning and its methods showed an insight to how much computers could see and understand what its seeing. However, acquiring language in the visual world would require computers to be able to communicate what it sees. Otherwise, in order to say that a computer is visually or linguistically intelligent one should imagine the computer having to pass the Turing test in a visual surrounding. 

% NI 21-09-27: describe all papers in order, starting from earlier work to later work (2011 - 2017)
Researchers attempt to improve systems that are capable to hold a dialogue with a visual content. \cite{das2017visual} trains a system in  encoder-decoder model on a data set of 2 pairs dialogue with an image content.\cite{Skoaj2011ASF} trains a system on learning concepts with visual content in an interactive-learning approach. In the similar context of improving systems that are capable of having more natural interactions, we see example in \cite{Lin2014VisualSS} of a VideoQA. 

% NI 21-09-27: This paragraph is here without any support. What do you mean when you say that we see less research on visual dialogue? That is not true. VQA is somewhat less complex then VisDial, because VQA is just a bunch of QA pairs, while in most of the VIsDial datasets you require memory and knowledge of discourse for the future questions. Also, in VisDial you do not have only question-answers following each other, there could be other elements - repairs, disfluencies, etc. The references to mention here: https://arxiv.org/abs/1907.05084, https://www.aclweb.org/anthology/2021.alvr-1.7.pdf, you can also find some work from CLASP people (Staffan must have something).
To make a true statement about the computer's capability to engage in a visual dialogue, it must be first ensured that the computer actually understands the questions being asked to it. Otherwise, dialogue is very complex with many elements determining its succession. In a dialogue with visual content, the computer must, furthermore, understand the questions within their visual context. It is reasonable that we see increasing research on "Visual Question Answering" and less on visual dialogue as a whole. Improvements in VQA intuitively means that we are moving closer in the direction of having an interaction with a computer in a visual dialogue.  
% NI 21-09-27: Embodied agents are expected to not only simply ask and answer questions, but engage in a conversation with humans. This means, that they can also produce a much bigger range of sentences - clarification requests, etc. I think your thought here is leading there. Or, at least, it should be leading there in the context of this section.


 \cite{VQA} is the first notable data-set published for Visual Question answering (VQA). The data-set consist of open-ended and free-form questions. The data contains 250,207 images from MS COCO \cite{lin2015microsoft} and other abstract scenes.The question types in the dataset require a range of different capabilities such as common-sense reasoning, knowledge-based reasoning, object-detection and active recognition.
 
 % NI 21-09-27: Make the idea of human workers writing question stronger. I think you want to say that these datasets are created by humans (unlike the EQA one, which you will describe later). Well, except the ren2015exploring. Start with saying that some datasets are artificial, and some are not. Give examples. Then say why synthetic datasets are created in the first place: they are cheaper, they are simpler to create, while it is harder to make humans ask questions and produce data that you need. A good reference and reading here: https://aclanthology.org/2020.coling-main.551
 Data-sets that use MS COCO scenes such as \cite{gao2015talking}, \cite{yu2015visual} in addition to \cite{VQA} used human workers to write the texts for the scenes. Other data-sets are generated automatically such as \cite{ren2015exploring}. 


\cite{zhu2016visual7w} introduces a  unique QA data-set. The Visual7W consist of questions about an image with objects marked with regions in the image. Object grounding with image region introduced in \cite{krishna2016visual} contains the largest data-set with regions for both VQA and Image-captioning. Object-region approach is intended to improve visual grounding, by marking the regions of the image that the strings refer to.

%restricted visual Turing test to evaluate visual understanding. The DAQUAR dataset is the first toy-sized QA benchmark built upon indoor scene RGB-D images. Most of the

% NI 21-09-27: In general, the sentences above need more connectivity, you can say more on these topics. I would suggest to restructure the text and make transitions smoother. We need more structure here (from captioning to vqa to visdial, then the point about real/synthetic). Embed literature as well.

\subsection{Embodied Question Answering}

\begin{figure}[H]
\includegraphics[scale=0.4]{images/EmbodiedQuestionAnswering.png}
\caption{The Robot is asked a question at a start position. It needs to look around, collect information and decide on the next step to take. When it recognizes the car, it stops and processes the scene to answer the question }
\label{fig:EQA}
\end{figure}. 

% NI 21-09-27: when citing multiple papers, just use comma between them: \citep{habitat19iccv,szot2021habitat}. \citep will put your references in brackets, not need to make your own brackets. It seems like \cite would put things without brackets, so you can use it as replacement for citet. But citep/citet is generally preferred instead of citep/cite.

Embodied Question Answering (\cite{embodiedqa}) \footnote{Link to the official page of EQA. It also includes other published papers about the task \url{https://embodiedqa.org/}.} is a new interactive task presented as one of the tasks within the Habitat Platform \citep{habitat19iccv,szot2021habitat}\footnote{Github link to the Habitat Platform. Information and code about EQA and the other tasks within Habitat can be found there \url{https://github.com/facebookresearch/habitat-lab}.}. The idea of the task is to allocate an agent at a random position in a 3D environment and ask it a question. To answer the question, the agent must intelligently explore the environment, collect information, and successfully navigate to the entity in question. EQA system navigates based on common reasoning, through an egocentric view, more or less imitating humans, it should be able to answer itself the common questions of "where am I?", "where to go next?" and if asked a question about the car, as seen in \ref{fig:EQA}, it should be able to reason that cars are usually situated outside or in the garage and look for the exit. Once it navigates successfully to a point where it recognizes the car, the robot should stop and answer the question.  


\begin{figure}[H]
\centering
\includegraphics[scale=0.3]{images/Vision-language.png}
\caption{EQA in relation to other vision\&language multi-modalities}
\label{fig:multimodal}
\end{figure}


In figure \ref{fig:multimodal} we see where EQA stands concerning other vision-and-language multi-modalities discussed earlier. In the language domain, we see single-shot and dialogue. VQA is a typical example of a single shot interaction, where the system is designed to take a single shot question and a visual scene and output an answer. On the other side of the language domain (dialogue), we see Visual Dialogue, where the interaction within a visual context is continuous. On the vision domain, we see that VQA Visual dialogue is distinguished from VedioQA and EQA by the visual input type. The robot in EQA is continuously moving while navigating, so it inputs the vision similar to videos. Finally, EQA is distinguished from the rest of the multi-modalities on the action domain by being active. Hence, action here refers to executions of commands in a physical space. EQA is action-active by its navigational functionality. The rest of the modalities are passive with no functionalities of physical action execution. 
% NI 21-09-27: Overall, what are the benefits of EQA? What it provides that other tasks do not provide?

% NI 21-09-27: Below you are describing novelty in terms of navigation, but our focus in on question answering. maybe you can combine this paragraph with the previous one and say that the novelty of the eqa task is shaped by both navigation and question answering. However, most of the research has focused on navigation tasks in this setting, you can cite other works by EQA guys here. What is missing is a deeper analysis of how system deals with question answering task. One recent addition towards improving QA is the following dataset - https://arxiv.org/abs/2011.08277, you should also include it in the background discussion here.
The novelty of this system is that it presumably solves the problem of navigating and performing tasks in unseen environments. Many of the earlier studies that deal with navigation, such as \cite{kruijff2007situated},\cite{lauria2001training} require the system to have a localized map of the environment to be able to navigate in it. The problem of localization in robotic navigation is known as Simultaneous Localization and Map Building(SLAM) problem. SLAM is a problem where a robot should map an unknown environment without a GPS or local map. Simultaneous localization is when a robot discovers it is surrounding and simultaneously construct a map while aware of its changing location. This means that the robot should extract information from its surroundings and learn the map as it goes \cite{grisetti2010tutorial} \cite{938381} \cite{8482266}. 

% NI 21-09-27: this paragraph seems out of place, should it be moved above when you talk about EQA?
The answering system in the robot consists of two core components. The first is navigation, and the second is Visual Question Answering. In principle, the task should be performed in conjunction between the Nav and the VQA model. The navigation should lead the robot to a correct viewpoint then freeze its move. The VQA model should then take static image frames of the scene from the viewpoint where the Nav stopped and answer the question. 
However, the system's design allows it to exclusively perform either navigation or visual question answering on baseline models. The ability to train and evaluate either of the modules is possible due to two different training setups.

% NI 21-09-27: why is this in background? Background introduces main concepts shortly, you have already introduced what EQA is. The details below should be a whole separate section about how to train your own EQA system.
\subsubsection{Training setups}

The first setup is a connected system with training in Reinforcement Learning setup. % NI 21-09-27: if you describe the training procedure, do it from the start. What is the input to the model and how is it trained? It is not immediately trained with reinforcement learning, first, it learns to imitate the golden paths. Only then, the navigation is fine-tuned with RL.
The training of the robot in RL happens based on the answer-based evaluation. The robot is rewarded if it completes the whole task using the two components connected.  The basis of evaluation in the RL setup is the answer prediction. The system is rewarded if it answers the question correctly, and to answer the question correctly, it needs to navigate to the right place and stop at a good view position so that the VQA system could have a relative and informative visual scene in order to answer the question. However, the researches in \cite{embodiedqa} elaborates that the system performs poorly when trained combined in RL. The navigation in the RL setup tends to position itself inaccurately at the stop-goal, which leads to passing distorted images to the VQA model. "Noisy or absent views" would confuse the question-answering model \cite{embodiedqa}. For the mentioned reason, there is no available RL-based system available for developers. 

% NI 21-09-27: ah, I see, you talk about it here. This part should be the first part.
The second setup is a system with the Nav and VQA components trained separately and differently. The navigation is trained in the 'Imitation Learning' setup, and  VQA is trained in Supervised Learning.\cite{hussein2017imitation} describes Imitation Learning as learning with a teacher, where a robotic system has to mimic the steps taken by its tutor. Imitation Learning is considered an effective solution, in particular, for navigational problems as its step-to-step learning restricts the freedom of systems; We see IL popular, for example, in navigational systems of ground vehicles \cite{silver2008high}. The available Nav and VQA models that are available for training and evaluation in the habitat platform are the baseline models.  The details of the training and the data used in each component will be described in more detail in the coming sections.
The main point we attempt to convey here is that the answering system being researched is one with Nav and VQA components trained and tested differently. % NI 21-09-27: not sure about this last sentence. I think what you are trying to describe is just to inform the reader about the training process of the agent (with imitation learning and RL). Just follow the paper and combine the two paragraphs you already have. There is no way we test QA system, which is trained different from how it is done in the original paper, so no need to somehow make it the main point. Just say that the nav system has been fine-tuned and also explain why fine-tuning was needed (it's in the paper).

\subsubsection{Data} 

% NI 21-09-27: an example of the room map from matterport would be good here, something similar to what they have in the original matterport website main page

The dataset for the EQA task is called "EQA-MP3D," and it is a synthetic dataset generated automatically.\footnote{The dataset can be found on the Github page of the Habitat Platform, attached in the main page in the section' Task Data-sets' \url{https://github.com/facebookresearch/habitat-lab}.}  The EQA-MP3D task dataset is applicable for navigation and VQA, meaning that training the navigation and VQA use the same dataset.  We refer to each question-answer in the EQA dataset as an "Episode" because each QA sample includes a complete trainable navigation episode. We could describe the QA episode as a function executed in a 3D environment to yield an answer. 

The 3D environments used in the task are indoor environments from the Matterport 3D(MP3D) dataset \cite{eqa_matterport}.\footnote{ The GitHub reop of the Matterport 3D \url{https://github.com/niessner/Matterport}.} The MP3D is 3D constructed scene dataset which contains 90 segmented houses. The EQA trains the robot in 57 MP3D environments and tests the robot in 10 other unseen MP3D environments.  

\paragraph{Data in the Navigation training}

In each QA episode, the information mainly used for navigation is a question,  an ID for the 3D environment, a unique starting position, a destination goal, and a path to the destination. % NI 21-09-27: if you describe input this way, you need to use math for this. $s_i$ is a unique starting position, etc...it is somewhat conventional/useful/convenient/better to put math-based things here together with words.
The mentioned navigational information, excluding environment ID and question,  are all represented in frequencies of coordinates. The starting position indicates where the agent should be spawned relative to the given environment.  The path is the shortest path that the agent would take to reach the goal, consisting of steps and rotations. The shortest path is data used for Imitation Learning as the robots have to imitate the steps found in it. The goal is the stop point that marks the end of the episode. The stop point of the navigation is the viewpoint of the entity in question. 

\paragraph{Data in VQA training}

In each QA episode, the information used for training the VQA model is an ID for the 3D environment, a question, ground truth answer, and the position of view. % NI 21-09-27: same as my last comment - when you talk about the input (what is used for training), you need to make it more formal with all mathematical notations etc, something similar to what you have when you describe input to the navigator down below
The mentioned information is automatically taken in a code part of the Habitat platform and reconstructed into a conventional  VQA dataset, QA pair, and a visual scene. The visual scenes are extracted using the view positions given in each QA episode in the corresponding 3D environment represented by the ID. 

\begin{figure}[H]
\centering
\includegraphics[scale=0.5]{images/VQAConstruct.png}
\caption{Locations of viewpoints of the entity taken from an EQA episode to extract a visual scene. The visual scene is then constructed with a QA pair to form a VQA sample of 5 frames of images, question and ground truth }
\label{fig:vqaconstruct}
\end{figure}.

The extracted scenes for VQA consist of 5 frames images taken from the viewpoint where the navigator is supposed to stop. In figure \ref{fig:vqaconstruct} we see an illustration of the structuring of the VQA dataset using the EQA task dataset (EQA-MP3D) and the scenes in Matterport 3D. The resulted VQA for VQA training is a Question-Answer pair with a visual scene. 

\paragraph{Data-set size \& Question types}
% NI 21-09-27: I see that you are giving explicit description of the EQA task with all these paragraphs. Make it a separate section, right after Background.
The question-answer data set contains three types of questions. Each question type is generated in a string template. The templates are as the following:

\textbf{- color\_room} template: "what color is <obj> in  <room>?"In these questions, the agent needs to find the room in question, look for the object, and answer the question. For the agent to be successful at reaching its target, it needs to know the difference between rooms, and objects, by implicitly recognizing that a certain room is a living room and not a bathroom and such.  

\textbf{- color} template: "what color is <obj>". The difference between "color" type and "color room" is that no room is specified in the "color" type of question. In the "color" type, the agent needs to figure out where to look by itself. For example, "what color is the fridge?" the robot needs to implicitly figure that the fridges are usually in the kitchen and navigate to the kitchen to answer the question. In other cases, the object could be in the vicinity of the robot's starting point so that all it needs to do is to look around. 

\textbf{- location} template: "What <room> is the <obj> located in". 

In EQA-MP3D, each object in a question is unique to the room. The latter means that for an object to be selected for a question, there needs to be only one instance of that object existent in the room. The reason for this is to avoid ambiguity and not to confuse the agent if there happen to be more instances of the same object in the room. 

There is a total of 11496 question episodes in the train split and 1950 question episodes in the "Val" split. As seen in figure \ref{fig:questioncount}, in the train split, there are 1830 episodes of "color" type, 8031 episodes of "color room," and "1635" of location type. For the validation split, there are 1335 "color room" questions, 345 "color" questions, and 270 "location" questions. 


\begin{figure}[H]
\centering
\includegraphics[scale=0.45]{images/Train.png}
\includegraphics[scale=0.45]{images/Val-set.png}
\caption{Number of question-answers represented by their types in the Train and Validation set}
\label{fig:questioncount}
\end{figure}

However, the number of unique question-answers is different from the number of episodes—a unique question-answer as a question-answer of the same strings and visual scene. For every unique question-answer and goal (scene), there are 15 different starting positions and shortest paths for the robot to train on for navigation. This means every unique QA in VQA is repeated 15 times. For example, in the validation set, the number of unique questions (same QA and goal-scene) of "color" type is 23, we multiply it with 15 (the number of starting positions for every unique goal), and we get 345, the number of episodes for "color" type in Val-set as seen \ref{fig:questioncount}. In the train set, the number of unique visual-question-answer for "color\_room" is 536, for "color" is 122, for "location" is 109. In the Val-set, the number of unique visual-question-answer for "color\_room" is 89, for "color" is 23, for "location" is 18. 
% NI 21-09-27: here, things like ``color\_room'' should be formatted differently. Use italics or texttt, for example.


\paragraph{Data Bias}

In all color questions (color \& color\_room) in the train set, we observe a total of 153  unique textual references. % NI 21-09-27: 153 unique answers? what are 'textual references'? Do not make complex notations, you can just say that, for example, different sofas in different questions do not refer to the same sofa.
A 'reference,' in this example, is a string that can refer to a specific entity in a specific or non-specific space. For example, 'sofa' in questions like "what color is the sofa?" is one reference. "sofa in the living room," like in color\_room questions "what color is the sofa in the living room?", is a second reference. "sofa in the bedroom" like "what color is the sofa in the bedroom?" would be a third and different reference. In order to gain insight into the data, we normalized % NI 21-09-27: this is not normalisation, just explain better what have you exactly done. 'reducing to a particular reference type' - what does it mean? You counted all unique sofas (depending on where they appear, e.g. living room, bathroom, etc) and examined the distribution of answers for questions about these sofas? Unclear a bit.
all the color questions by reducing them into questions of a reference type and collect the number of answer choices found for each reference type in all color questions.\footnote{ Link to the statistical analysis of the data in a notebook  \url{https://github.com/Al-arug/EQA}.} 


\begin{figure}[H]
\centering
\includegraphics[scale=0.5]{images/AnsRef.png}
\caption{The color questions in the training data sorted by textual reference. In total we find in all color questions 153 references.
% NI 21-09-27: you need to give an example of these 'references'. It is unclear what you mean with them. We need to talk about these references more.
22.4 percent of the references have one color answer as the only choice.  }
\label{fig:AnsRef}
\end{figure}.

We find that 22.4 percent of the references have one color answer as the only choice of answer, as seen in figure \ref{fig:AnsRef}. This means that 22.4 percent, around 2208 of all the  9861 "color" \& "color\_room" QA episodes in the train-set, have one possible color as an answer. This means, in these cases, the model does not train on disambiguation any classes of color for the reference in the question.
% NI 21-09-27: this means that there is typically a single colour as an answer, and the model might learn this bias because of such frequency. Is it correct? Say it a bit simpler.
Instead, these data samples would tell the model that there is only one possible answer to memorize for this reference. (Appendices contain all the textual references found in the question)

\begin{figure}[H]
\centering
\includegraphics[scale=0.5]{images/biasperRef.png}
\caption{}
\label{fig:biasRef}
\end{figure}

% NI 21-09-27: this is very similar to what you said before...I am confused, we need to talk about these biases.
The second type of bias is the dominance of one color over the other choices in the question-answers with multiple color choices. In \ref{fig:biasRef} we categorize references per answer-choice. References that has two answer choice in one category, references with three answer choices in one category, and references with 4 answer choices in a different category. The bias for each category is determined differently. In the references with category two answers, a reference is considered to contain biased QA if the answer in 75\% of the instances of the answer is the same . For the categories 3 answer choice and 4 answer choice, biased is considered if one answer made up 50 percent of the answers in each reference.  In total we get that 23.5\% of the references with two answer choices are biased. 51.2\% of the references with 3 answer choice are biased, and 17.5\% of the references with 4 or more possible answers are biased.


\subsubsection{Navigation Model}


Habitat's navigation is referred to as PACMAN. It consists of two core components, planner and controller. The planner takes inputs from the vision and language model, and the encoding of hidden-layer and action of the previous time-step then outputs action-decision. 

\begin{figure}[H]
\centering
% NI 21-09-27: center all your figures
% NI 21-09-27: also, while I remember, for math formulas and anything mathematical (h_t, etc), use \align environment and put them in the center. Do it if you want to put them as equation. I have seen in other parts of the thesis that you do it sometimes, but do it consistently.
\includegraphics[scale=0.53]{images/nav.png}
\caption{}
\label{fig:nav}
\end{figure}

The controller takes the previous hidden state and action decision and executes the action. As seen in \ref{fig:nav}, visual input is passed to the control then the controller classifies the following decision of two possible decisions. Either to repeat the last action given by the planner or to return to the planner. The controller can repeat the same action maximum of five times then it automatically returns to the planner. 

Visualization of the navigation is in figure (1). T stands for the planner's time-steps, t = 1,2,3...., and N(t),  n = 0,1,2,3.. denotes the controllers time-steps. The denotations of symbols explained clearer in the quotation : 

% NI 21-09-27: use $$ for math environment, not \begin{math}.

"\begin{math}  I_{t}^{n} \end{math}denote the encoding of the observed image at t-th planner-time and n-th controller-time. The planner is instantiated as an LSTM. Thus, it maintains a hidden state \begin{math} h^{t}\end{math}
(updated only at planner timesteps), and samples action 
\begin{math}  a_{t} \ \in \ \{forward,\ turn-left,\ turn-right,\ stop\} \end{math} "p(6)
\vspace{0.3cm}

For example, the first  step-decision from the planner is denoted as such: 

% NI 21-09-27: this should be an equation in align environment
\[ a_{t} ,h_{t}{}\leftarrow PLNR\left( h_{t-1} ,I_{t}^{o} ,Q,a_{t-1}\right) \]
        


The planner computes the next step-action  \begin{math} a_{t+1} \end{math} from input of the previous hidden layer \begin{math} (h_{t-1}) \end{math}, question encoding (Q), the previous action  \begin{math} a_{t-1} \end{math}, and the image input given to the PlNR \begin{math} (tI_{t}^{o}) \end{math}.The planner selects the action \begin{math} a_{t+1}\end{math} and update the hidden state \begin{math} h_{t+1} \end{math} then passes the control to the controller. 




The controller decides to either repeat the action or return control to the planner. The controller's classification is based on the current hidden-state\begin{math} h_{t}  \end{math} and current action \begin{math} a_{t} \end{math} and the image observation from the planner + the image given at the controller's time-step. The denotation of the classification is as such: 

 % \left h_{t}^{n} \right \backepsilon

\[ \{0,1\}   \ni \ c_{n}^{t}  \leftarrow CTRL \left(h_{t} ,a_{t} ,I_{t}^{n}\right)  \]

"if \begin{math} c_{n}^{t} = 1 \end{math} then the action \begin{math} a_{t} \end{math} repeats. Else \begin{math} c_{n}^{t} = 0 \end{math} or a max of 5 controller-times been reached, control is returned to the planner"p(6). The \begin{math} h_{t} \end{math}   \begin{math} a_{t} \end{math} coming from the planner act as an intent. The controller, initiated  as "feed-forward multi-layer perceptron with 1 hidden layer",repeats and controls the action in order to align \begin{math}  I_{t}^{n} \end{math} with intent given by the planner. 

\subsubsection{VQA Model}



The VQA model is a CNN-LSTM architecture. The CNN encodes 224x224 RGB images with a "multi-task pixel-to-pixel prediction framework" (p6) encoding. The structure of the CNN4 {5x5 Conv, BatchNorm, ReLU, 2x2 Max-Pool blocks}, and they produce a fixed-size representation. "The range of depth values for every pixel lies in the range r0, 1s, and the segmentation is done over 191 classes" (p.11) \cite{embodiedqa}. The "lstm" is a 2-layer LSTM with 128d hidden layers.





\begin{figure}[H]
\centering
\includegraphics[scale=0.35]{images/VQA.png}
\caption{Architecture of the VQA model consist of and LSTM for language encoding, CNN for vision. The system is trained to combine the two with attention}
\label{fig:VQ}
\end{figure}



The CNN extracts features from five images(5 frames) scene, and the LSTM encodes the textual features of the question. Combining the visual and linguistic features is done through computing similarity via dot product and concatenation. First, the similarity between each image and the question features is computed via dot product. A softmax converts the question and image similarity into attention wights then the question encoding is concatenated with them. The concatenated features are then classified in a softmax, where the answer probability is distributed over 172 answers.(\cite{embodiedqa},p6)



\subsection{Problem}

In an experiment we conducted on the VQA model, we observed that the system tends to answer the questions relying mainly on the textual input in the questions (bias)\footnote{ Link to the experiment "Testing VQA's reliance on vision "\url{https://github.com/Al-arug/Habitat-Project}.} The idea of the experiment was to give the model a random image instead of the original scene and see if it affects its predictions. The results showed that the system gave correct answers despite the absence of the corresponding scene required to answer the question. In such a case, the system's performance would typically have worsened, not improved, as the required visual information to answer the question is missing. The correct answering by the system was demonstrated in an overall increase in the performance score. Its ability to answer correctly demonstrates its reliance on the language model to predict the answer. 

The system's ability to predict answers correctly in the experiment indicates a lack of visual grounding. We draw this conclusion from the observation that vision did not influence the predictions. This means that the system, in training, has not learned a scheme for word-meaning in association with vision. Grounding language in vision is when we connect the "high-level" symbolic representations such as language to a "low-level" non-symbolic representation such as the sensory (visual) features. The ability to ground language in vision is essential for any task requiring "seeing" and attending answer. If a robot successfully learns to align and combine the two types of representations, one could say that the computer understands what it sees (visual grounding). When a system fails to achieve such a connection, we define the problem as the "Symbol system problem" (\cite{harnad1990symbol}) or 'lack of visual grounding.' 


We presume that the lack of visual grounding is attributed to bias in the dataset. Earlier in this text, we reviewed textual biases within the EQA dataset. Having biases in the dataset would hinder the learning process, as it gives the model a way to learn to avoid combining vision and language by giving correct answers without actually learning to combine the two types of data. 

We also observe that the type of questions asked are simplistic and can be considered unnatural. The existent color questions in the dataset are not the type of questions that a human would naturally ask. The limited types of questions found in the dataset seem to be meant to simplify the robot's task with a primary focus on navigation.


\subsection{Problem in a context}

(\cite{selvaraju2020squinting},\cite {goyal2017making}) and other research within the VQA point out the problem where models learn biases in training and manage to give good results in the testing.\cite{johnson2017clevr} elaborate that the underlying issue here is that the model answers by memorizing prior textual information. For example, a neural network might answer the question "What covers the ground?" correctly by answering "snow," "not because it understands the scene but because biased datasets often ask questions about the ground when it is snow-covered." \cite{fukui2016multimodal} clarify that the models' answer-cheating is demonstrated when a VQA system primarily relies on the language model and ignores the visual information. Such a learning problem is crucial because it makes it challenging to evaluate the model's improvements\cite{agrawal2018don}.

When a system cheats its way into answering the questions, it shows a lack of visual grounding\cite {goyal2017making}. Visual grounding (understanding the meaning of words about vision)is crucial because we want the systems to understand the reasoning steps that humans would logically take to answer a question \cite{agrawal2016analyzing},\cite{zhang2016yin}, \cite {fukui2016multimodal}. For the system to be able to reason its way to predict an answer, it must first capture the full meaning. \cite{selvaraju2020squinting} explains that learning to reason would require the systems to make inferences at "multiple levels of abstraction." For example, "is the banana ripe?" where it would instantly answer "no."  Answering this question would require the system to rely on perception to answer sub-questions such as where is the object? What are its shape, size, and color? Then reason that the "yellow" color indicates ripeness.\cite{selvaraju2020squinting}

\subsection{Research Questions}
\begin{itemize}


\item How can we extend the dataset with more sophisticated and natural questions? (A useful robot should answer a variety of questions.)

Adding new questions could help test the system's capabilities, but more importantly, we consider it a step to enhance the system's cognition. The VQA system that we are improving is part of a robotic system that should ideally be helpful for human use. Social robot's usability is very dependent on its exhibition of human intelligence \cite{fong2003survey}.

\item How does the VQA system perform with the new question types? 

\item Does asking questions of spatial and size types improve the system's attention to vision? (Evaluating it based on the performance on color questions) 


\end{itemize}
 



\newpage

\section{Methods and materials (proof reading required)}
\label{sec:methods}
\subsection{Habitat Simulator and Lab- Overview}

We use the Habitat platform as a host for the EQA task. The name 'Habitat' is derived from  the notion of learning within and from an environment. Imitating our natural habitat, the Habitat platform facilitates spawning an agent in a simulated environments with the possibility of teaching the robot to preform different tasks. 

The perquisites needed to test or train an agent for a certain task in a given environment, are facilitated by a core component called  Habitat Simulator. Habitat Simulator is responsible for simulating an  environment and insinuating a robot in it. The simulator acts depending on the configurations given to it. 

The configurations are processed into commands in Habitat-lab before being passed to the simulator. Habitat lab is the second core component of the system. In addition to giving commands to the simulator, the Habitat Lab module acts as a pipeline that prepares the data-set of the corresponding task. The habitat-lab module,in other words, is the coordinator that informs the simulator of the required setting, and the data loader and processor that prepares the data for either training or testing. 

\begin{figure}[H]
\centering
\includegraphics[scale=0.43]{images/configProcess.png}
\caption{Example of Habitat lab processing the configurations to implement validation for the VQA model }
\label{fig:configs}
\end{figure}

Figure \ref{fig:configs} resembles a map of the code structure when the habitat lab module is initiated to preform validation task for the VQA. Each task has its own configurations and in this example the task is 'VQA evaluation'. As seen in the figure \ref{fig:configs}, the module takes hierarchical steps in which each step is executed in accordance to the configuration of the given task. In the most down box of the structure we see parts of the commands directed for the simulator, such as insinuating an environment and sensors in the agent. Other commands include registering a data-set which takes part in lab module. 

\subsection{vision}

The vision of the system relies on egocentric 224x224 RGB images processed in CNN. The CNN encoding has the functionality of a “multi-task pixel-to-pixel prediction framework,” which consists of 4 {5x5 Conv, BatchNorm, ReLU, 2x2 Max-Pool blocks}, and they produce a fixed-size representation.“The range of depth values for every pixel lies in the range r0, 1s, and the segmentation is done over 191 classes”(p.11). (page,6). 

It  is possible to train the encoder-decoder on generating  three sensory information. The three decoders, which can also be referred to as sensors take the functionality of: 1) RGB reconstruction, 2) semantic segmentation, and 3) depth estimation. The latter sensors are used to obtain “object attributes (i.e., colors and textures), semantics (i.e., object categories), and environmental geometry (i.e., depth).” . 

In the baseline models, different tasks take different sensors.Not all the above-mentioned sensors are used in all the baseline tasks. Since navigation and VQA are trained and evaluated separately, we refer to them as separate "tasks". The two tasks in EQA take the following sensors: 

{Navigation}:  "depth" and "RGB". Depth sensor is essential for the agent's capability to navigate. With depth sensor it could estimate distances and avoid colliding with obstacles.  

{VQA}: The existent baseline VQA model uses the visual information with "RGB" data only. (No reason mentioned to why the other sensors are not used in the question-answering baseline module).  



%Hence- current hidden-state h_{t} and the current action are only %updated in the planner: 
%
%The controller executes the action  a_{t+1}. It takes then the 
%
%Hence, h_{t-1} and a_{t-1} carry no information at the 0 timestep %since no encoding or action been outputed by the planner at the 0 %step; Thus they are more like an initiation in this example.  
%
%The planner passes the current action a_{t} and the current hidden %state h_{t} to the controller 
%



\subsection{Data and Data-sets}

 Our method of generating questions is largely related to the structure of the data-sets in the initial EQA paper\cite{embodiedqa}. Understanding each part of the data-sets and their structure would give an insight into the work flow of generating questions to the task. In this section we elaborate on the source of the data-sets, their content, and our methods in processing the data. 
 

\subsubsection{Semantic annotations in Matterport}

%(restructuring is required-- more precision) (examples to rephrase-- why do we need the %location in global coordinates and why the camera views are also important)
\paragraph{Annotations}
In the Matter port annotations, Each house environment comes with three files. The three files are x.house,x.ply and x.. We collect the annotations from the x.house files house. 

Each house file comes with eleven line-types of annotations.\footnotetext{https://github.com/niessner/Matterport/blob/master/data\_organization.md}. The lines are marked by a capital letter as a marker; the first letter-marking to the last letter are as in this list [H,L,R,P,S,V,P,I,C,O,V]. Each letter-marker symbolizes a certain type of information. In this section, I am going to explain only the type of information that we use in this project.

The only data we extract from the house file, is the "O". The "O" lines contain information about the objects in the house. Every line that begin with an O letter consist of one object in the house with a corresponding information about its geometry and location within a room and level-floor. Each "O" line looks as such: [ O object\_index region\_index category\_index px py pz  a0x a0y a0z  a1x a1y a1z  r0 r1 r2 0 0 0 0 0 0 0 0 ] 

The data of the object in the line seen above comes in a string form, and each section in the string represents different types of information. \textit{Object\_index}, the index of an object is what we refer to as the object ID. \textit{region index} is the room ID. \textit{category\_index} is the object's index in category map; this index is used to obtain the object's name from the category map.\textit{px py pz} represent the center of the box in (x,y,z) axis. \textit{a0x a0y a0z  a1x a1y a1z} these are the rotation of the OOBB and AABB. \textit{r0 r1 r2} represent the radius of the object from the center on the (x,y,z). Finally the last "0"s in the line have no meaningful value, and therefore are ignored. 

\paragraph{Views of the geometric data}

 The geometric information consist of elements as location of an object, region or level, defined by their center in a world coordinate system, as seen in the previouse section. Other information is the size of the entity given its radius from its starting location (center).   

The camera views of the scenes are globally oriented \cite{Matterport3D}(p3). A way to allocate an object is to find its location in a accordance to global coordinates. Let's say the global coordinates start from the center of a house where the center of the house is (0,0,0) on the (x,y,z); and let's say all the objects are spawned through out the house's (x,y,z) axis where each objects location is defined by its distance to the house center. When annotated, the objects are viewed through a camera. The description of their geometric location, thus, should consider the view-postion of the camera. 

\begin{figure}[H]
\centering
\includegraphics[scale=0.53]{images/campos.png}
\caption{The camera in graph A views the objects from global perspective(readers perspective). The view of the camera in graph B is rotated. The rotation is resembled in the the axis's representation }
\label{fig:campos}
\end{figure}

In graph (A) in \ref{fig:campos}, we see that the camera-view of coordinates align with the global coordinates.The (x,y,z) that go through each object in graph(A) and graph (B) are the view of the axis in reference to the camera. However, if the camera is positioned to the right of the object from our view, as in graph (B), then we say that the camera view of coordinates is not aligned with the global view. We notice in graph (B) that from the camera view, the "global X" is "Y" and vice versa.

Some geometric calculations cannot be preformed if the location measurements are not relative to each other. For example, if we want to calculate the distance between objects the locations must be consistent with one reference point. The camera position is changing and if the location of an object is referenced by the camera's position then we would get locations relative to the changing position of the camera in a time-span. 

To globalize the orientation of the view, measures such as top-down view of a map, or calculating the rotation of the camera from the global center. While the global locations are crucial for measuring the distance, other point-views are also crucial for other purposes. There are three essential coordinate systems to know when working in a 3D environments: 

\textbf{1. World coordinates}(global):  World coordinates(global): The coordinate system that starts at the center of the world; a house in our example. The center of an object in this coordinate system, is then decided by its distance to the center of the world. 

\textbf{camera-view coordinates:} The coordinates from the camera's views. The center of this coordinate system is the position of the camera. The center of the object in this world is defined by its distance to the camera. 

\textbf{3. Local view:} The center of the local view is the object itself. 

The center of all these views is (0,0,0). We described above that the world coordinate system allows us to measure distance between objects in a world map. The camera view is useful if a robot is expected to navigate an environment and describe spatial relations between objects such as "next to", "above". The local view could tell about the size of an object. In particular, the (x,y,z) from a local point of view tell about how far the object stretches from its center where the center is (0,0,0). The local view can be referred to as "radius". 

MatterPort 3D provide the views decribed above. We discuss in more detailed the usage of the object's location in global coordinates and the local view in details in the implementation part.  

\paragraph{Processing geometric data}

\begin{figure}[H]
\centering
\includegraphics[scale=0.5]{images/A-OB.png}
\caption{2D AABB represented in blue square with its axis aligning with the world view of axis. 2D OOBB represented in red square has its axis rotated from the global view }
\label{fig:A-OB}
\end{figure}


The geometric information can be classified into two main categories. In a 3D environment we can imagine each object having  a labelling box rotated and oriented around its shape, and other box that bounds the first box with the global coordinates. In figure \ref{A-OB} we see a demonstration of the two boxes in 2d squares. The red box, that is meant to surround an object, is referred to as 'Object Oriented Bounding Box' (OOBB). The coordinates of the OOBB are rotated with rotation of the object (rotated in accordance to the local view). The blue box is referred to as 'Axis Aligned Bounding box'(AABB). The axis of AABB are aligned with global coordinates. 

We extract and save one specific information type of each box. The feature we take is the "radii", or else can be referred to as half-extents. The half-extents (radius) can be helpful in representing the object in different ways.  

We use the radius of the OOBB box to calculate the size of an object. The volume of OOBB gives more precise estimation of the size of the object, as the box is more enclosed around the object.

We use the radius of AABB box to measure the distance of an object to other objects. We can measure the distance between objects in an environment by the distance between their centers. More precise measurements would be to measure the edges or the corners of the object. We get the corners by measuring how far the box stretches(given by radius) from the center.  

Important to mention that we locate entities on a map with coordinate points that are positioned in accordance to one coordinate system (grounded in a the global map). Otherwise the numbers that represent positions would be in-indicative of points in the global view . It is for the latter reasons, the centers (located globally) are helpful to measuring distance-- because they are located on the same coordinate basis.  

The AABB box, provide a straightforward estimation of the positions of the object's shape in the global map. The local view of the AABBs are aligned with world coordinates, therefore allocating its corners globally would only require an estimation of how its radius(given in alliance with world coordinates) stretches from the center.

True that the OOBB corners are better representatives of the objects corners, however, locating the oobb corners in the world map is not simply done by measuring how its radius stretches from the center in the direction of  the global axis as in AABB. The radius of the OOBB is given along its local view axis(its rotation), meanwhile the center point is given in the world axis. Thus, locating the positions of the OOBB, using the same radius-length measure, would require adjusting the radius to the direction of its rotation. Therefore, the global-alignment characteristic of the aabb provides a direct way to locating its edges.  


The corners of the OOBB would be the most precise representation of the corners of an object. 



\subsubsection{EQA (Task Dataset)}

Our method for generating questions relies on imitating the structure of teh EQA-mp3d data-set. In this section we give a review of the EQA-mp3d structure and attributes.   

\paragraph{Structure}

\begin{figure}[H]
\centering
\includegraphics[scale=0.5]{images/episode1.png}
\includegraphics[scale=0.5]{images/datasplit1.png}
\label{fig:episode}
\caption{}
\end{figure}

In figure \ref{fig:episode} we see the top structure of the val and test. \textit{Episodes} refer to each question-function in the data-set split. \textit{Question vocab} and \textit{answer vocab} contain the same elements as dictionary keys. The elements are: [word list,stoi,itos,num vocab,pad token].

"Question vocab" and "answer vocab" in the "train" and "val" are identical to each other. When using each split of the dataset, the answer-tokens that are considered are the ones contained within the episodes instead of the word-lists mentioned above. 

Each question-sample is an episode that consist of multiple layer information. The structure of one episode of all the "episodes" is as seen in figure(x). We describe the elements of an episode in the following:  

\textbf{House ID}: The house ID given by the house ids in MatterPort3D.
\textbf{Episode ID}: The episode index in the range of the split's length. \hspace{1.5cm}
\textbf{Info}: This element contains all the information about the the object and room in a question. The information is structured as such: 

Information about the traget-object is the first layer within "info": 

\textit{centroid}: The center of the object's box in the global coordinates. Box is the area that labels  the object. When the center is globally oriented we would refer to this center and box as Axis-aligned bounding box(AABB), which means that (x,y,z) axis of the center are aligned with global coordinates. 

\textit{radi}: It tells how far the box (object) stretches from its center one direction of each axis. The value of radii is relative to the object itself (from the local view), where the center is zero. If we have, for example a radi of (2,1,4), this means that the object's box stretches +2 and -2 from the center on the x axis. The boundaries of the object's box relative to itself is referred to as object oriented bounding box (OOB). 

\textit{level}: at which level-floor of the house  is the object located in. 

\textit{room-id}, \textit{room name},\textit{obj Id},\textit{room name} : Room ID, room name and object ID as given by semantic annotation in  Matterport3D. Many of the objects are re-named, mostly names in hyponymes changed to hypernym category such as: round-sofa, l-shaped sofa changed to their hypernym category "sofa". 

The second layer is information about the room: 

Information about the room is similar to the type of information given for the objects. Th information is \textit{floor-level}, \textit{room-id}, \textit{room name},

Final layer consist of a "question-meta" which includes the color of the object. This section also includes question-entropy ..... 

The elements that are marked in blue in figure(x) are navigation-related material. 

\textbf{start position}: The start positions are all unique. For each unique question in the data set there is fifteen different starting position. 

\textbf{rotations}: This is the rotations that the agent have to do while navigating. It stands as supplementary information for the shortest path 

\textbf{goals}: Goals are the destinations that the agent should reach in navigation. The goals stand for the possible view points from where the the target object can be looked at by the robot. Each view point consist of geometric position and the rotation toward the target object respective to the position. 

%\begin{figure}
%\includegraphics[scale=0.5]{images/datasplit1.png}
%\end{figure}









\newpage

\section{Task One- Question Generation (Final review required)}
\label{sec:task1}
\subsection{ Overview Extending Dataset}

The idea is to include more questions about the same objects and the scenes found in EQA-v1. The question generation copies the EQA data-set and modify the so called-episodes in it. The episodes, as mentioned in previous sections, are executable functions when inserted in an environment they yield an answer. The scenes denote the visual scene of the destination goal of the a navigational episode. The system learns to reach its navigational goal by a shortest path included in each EQA-V1 episode. Our new questions use the same shortest path found in EQA-V1. 

This project consist of two major module-components. The first module is a parser that does data extraction, and acts as a processor for raw data by transforming into usable geometric information for generating question-answers. The second module is the question-answer generator. This chapter describes the projects construct and the usage of each part of it.  



\subsection{First module- Data parser}

Our house parser consist of two classes. The first class is a class that parses the houses into a structural data. The second is functional class we use to find near objects close to a target object. The latter class is used 

The experiment include used two different ways for annotation extraction. The first source-method is the raw annotations given in the 'house files' of the MP3D data-set. The second method uses Habitat's simulator and sensors. The annotations extracted from the sensors in the Habitat's simulated environments provide more computed information and slightly different raw data MP3D annotations; In particular, some object names are different, but the rest of information, such as object ids and location-centers, is consistent with the annotation of the MP3D. 

In the existing generated question-answers data-set we use the data extracted from the Habitat semantic sensors. The main reason for choosing Habitat's semantic sensors is because they provide a computed geometric information of the objects such as the location of an object within an Axis Oriented Bounding Box (We elaborate on this term in the coming section). An additional important reason for this choice of extraction is that some of  objects names output-ed by the sensors are aligning with the names found in the original EQA-V1 data-set. For example, object names in MP3D such as l-shaped sofa and rounded-sofa are transformed, in Habitat's sensor, into their Hypernym category 'sofa'. Choosing object names that are aligning with names found EQA-v1, is helpful for having the overall data consistent with each other when we emerge our generated questions with EQA-V1. 

The second major component - get close distances 


\subsubsection{Annotations from MP3D files}


We extract two types of raw information from each object's line of annotation found in the house files in Matter-port. Lines mentioned erlier: [px py pz  a0x a0y a0z  a1x a1y a1z  r0 r1 r2 0 0 0 0 0 0 0 0 ] 

First we take the obj and room indexes (ids). Second is the [px py pz] and where we categorize it as the center of the object's box. Third is the [r0 r1 r2] (radius-half-extent).

we structure the data in a form that the annotation of a house begins with the first level in it, followed by the rooms and objects in each room as: house 1 [{level1:room1[bedroom]:(obj1:bed,obj2:..),room2:(obj..)},, {level2:......} ]  

\subsubsection{Annotation extraction using from Habitat's sensors}

Our final choice for extracting semantic annotations is Habitat's simulator. Our annotation's parser of the houses uses the sensors with configuration provided by the habitat platform \footnote{url{https://aihabitat.org/docs/habitat-lab/habitat-sim-demo.html#scene-semantic-annotations}.}. The configurations include the settings such as the scene, the height and width of the sensors, and the types of sensors to include. color sensor, semantic sensor and depth sensors are used. 

Once we simulate the environment, the sensors output the annotations of a house(scene) as an object. We iterate through the object to obtain information about the levels, rooms, and the objects in the rooms. We freeze the simulator once the annotations' object of one environment is outputted, then repeat the process for the other environments. 

The data about the an environment's annotations include a calculated geometric information. The sensors in the simulator gives an advantage by calculating the sizes of both the AABB and the OOBB. We use the sizes of the OOBB and AABB, to further obtain more specific information about the object's boxes. 


\begin{figure}[H]
\centering
\includegraphics[scale=0.5]{images/Geoinfo.png}
\caption{Min and Max of an Axis Orients bounding box}
\label{fig:aabb}
\end{figure}


We make calculations from the data we extract in order to obtain other necessary info for generating question. Important to mention, the geometric points extracted by habitat's simulator, such as centers of the boxes and the sizes of the boxes' sizes, are represented in 3D data where the  x is the length, y is the width, and z is the height. 

The first calculation is finding the 'min' and 'max' of a bounding box given an object's center and length of its sides on(x,y,z). In figure \ref{fig:aabb} we see visual illustration of the extracted and calculated data. The min represents the corner of a box that has the lowest value of (x,y,z) and max is the corner with largest value for (x,y,z). Other way to put it, the min represents the corner point in the minus direction from the center in all the axis , and max is the corner on the positive direction from the center in all axis. 
%$\displaystyle
The given size of the box is from the min point to the max point with a 3d value (x,y,z). To get each of the points we first get the half extent of the size such as: \begin{math} Half\ extent\ =(x,y,z) /2\ \end{math}. half extent or (radius) is the point from the center to either the min or max. 

The 'min' is the point stretched by the length of the half extent in the negative direction, and 'max' is the stretch to the positive as the following:   \\ 
\begin{math}
Min\ point\ =\ C\ -\ \vec{H} \ \ =\ ( x\_{1} +x\_{2} ,y\_{1} +y\_{2} ,z\_{1} +z\_{2}{}) \  \\
Max\ point\ =\ C\ +\ \vec{H} \ \ =\ ( x\_{1} -x\_{2} ,y\_{1} -y\_{2} ,z\_{1} -z\_{2}{}) \ 
\end{math}

The min and max help us in define spatial relations among objects. How we use them is explained in the coming sections.

We use the OOBB sizes for calculating the sizes of the objects. We consider the size as the volume of the box which is the length multiplied with the height and width. In our case the length is the x value, width is the y value and z is the height, then  the calculated volume of a box is \begin{math} X\ x\ Y\ x\ Z \end{math}. 


We structure the annotations and save them in a file. The structure of the data consist of a dictionary storing the data in a hierarchical way. At the top part is the house id, then rooms in the house, then the objects in the house. Each object stored by id, contains the min and max value of its box, size of its aabb, its name, room name and id where its located, and the level id where the room is located. Storing the annotations this way allows to access all the objects in a room through the scene id and room id. 

 We store the calculated volume of each object in all the houses and store it in a second file. The volumes of objects are stored by their object category. In the volumes file, we find the volumes of all the objects in all of the houses stored in a dictionary, each key represents a category such as 'sofa' with values of the volumes of this object type. The point here is to obtain data on the sizes of each object type. We use this information for finding ground truth answers about for the size questions.  
 

\subsubsection{Second class - Distance and size calculator}

The main functionality of this class is to find a spatial relation between pairs of objects in a room. It takes as an argument scene and room id and uses this information to access the objects in a room from the parsed houses files. 

It outputs three types of spatial relations between a pair of objects. The pairs spatial relation is specified by whether an object is 'on' , 'next' or in unspecified 'close' distance to a second object. The pairs are organized in a dictionary, one key for each spatial relation. This information is used for generating positive spatial questions. 

The spatial relations mentioned above are measured by calculating the distance between the corners of the objects' bounding boxes along certain dimensions. The corners obtained using the 'min' and 'max'. If we iterate over the Min(x,y,z) and Max(x,y,z) we get the other six corners of the box. Figure \ref{fig:box_points} illustrates the eight corners, the view point of the cube is rotated to the right for the sake of viewing all the points in the cube. If we move our point of view directly in front of the cube as if  we are facing the square GHED,  the points A and H would seem to be lying on a straight line. Lying on the same straight line, for example, means the point A and H are located on the same points in the x-axis, and so one for the other parallel points .\begin{figure}[H]
\centering
\includegraphics{images/cornerPoints.png} %[scale=0.5]
\caption{}
\label{fig:box_points}
\end{figure} 

We get the rest of points from the Min and Max of an AABB for the reason that AABB's are not rotated and aligning with the global view. To express it better, we image the global point of view of the AABBS as a view facing a group of adjusted and not rotated boxes. 

For our example in figure \ref{fig:box_points}, the values of the six corners of the box found from the Max and Min in addition to the corners of the Min and Max would be as such: 

$\begin{array}{l}
A\ =( x_{max} ,Y_{max,} Z_{max}) ,F=\ ( x_{min} ,Y_{max,} Z_{max}) ,H\ =\ ( x_{max} ,Y_{min,} Z_{max}) ,\\
B=( x_{max} ,Y_{max,} Z_{min}) ,D=\ ( x_{max} ,Y_{min,} Z_{min}) ,\ C=( x_{min} ,Y_{max,} Z_{min}) ,\\
G=\ ( x_{min} ,Y_{min,} Z_{max}) ,\ E\ =( x_{min} ,Y_{min,} Z_{min})
\end{array}$

  


A visual representations of the corners and their values seen in figure.X 


The definition of each of the mentioned spatial relation , is an approximation of how we define them as humans. Each of the spatial relations are determined given a geometric criteria of distances and positions between objects' corners along the three axis  (x,y,z). 

Two general operations are used in each of the relations. The first one is calculating  the Euclidean distance between two corner points; denoted as the distance between p and q in this formula:  \begin{math}
 d\left( p,q\right)   = \sqrt {\sum _{i=1}^{n}  \left( q_{i}-p_{i}\right)^2 } 
 \end{math}

The corners, depending on the type of relation we want to extract, can be represented as points in  1d, 2d, or 3d. 1D is  when we pick points from one of the axis only. 2D is a point on two axis, and 3D is a point on three axis as the corner examples illustrated earlier. We describe this in more detail in the sections below. The second operation is calculating if one of side of a box on a certain axis is contained within the other. 

\begin{figure}[H]
\includegraphics[scale=0.5]{images/contained.png}
\caption{}
\label{fig:contained}
\end{figure}

The calculation if one side is contained within the other relies on defined criteria. The calculation is done in a specific function and  can take as input  corner points on one axis or more. In the drawing \ref{fig:contained} 'A' represents two lines on the 'X' axis where the top part is not contained within the other, an in 'B' the top is contained within the lower line. In this example we determine the 'contain' relation by taking the 'min' represented by the the orange dot and the 'max' doted in red. 

If the min of the upper line is greater than the min of the lower line and the max of the upper line is less than the max of the lower line then the upper line is contained within the lower. Hence the 'min' of the upper line is greater than the 'min' of the lower line because it's more to the right in the positive direction of the 'x' axis. If the previous condition is not satisfied, then the lines are otherwise not contained. 


The operations above are done over different axis for every relation. Below we specify how each of the 'on', 'next to', and 'close' relation is determined between the objects. 

\paragraph{On}



First step is choosing pairs of objects that are closest to each other vertically(on the Z axis). The distance should be less than a 0.5 millimeter thresh hold. The Euclidean distance here is calculated between the 1D points on the Z axis only. Two corner values from each box and any corner match the distance required are picked out. 

\begin{figure}[H]
\includegraphics[scale=0.7]{images/on.png}
\caption{}
\label{fig:on}
\end{figure}


The second step consist of a group of conditions that the pair of objects need to meet in order to be considered on each other. The first condition is that the vertical sides, the line from Zmin to Zmax, of the boxes are not contained within each other. The Zmin-Zmax lines of every object box are the lines between the two red dots in box A ans B in the illustration \ref{fig:on}. Otherwise if the lines on the Z axis are contained it would mean one object is inside the other. 

The second condition is that the horizontal line of one of the boxes is contained within each other.The horizontal lines in the illustration are from the Orange to the red points in each box. The lines on the y axis from the red to the blue points should also be contained. 

The final step is deciding which object is on the top of the other. The pair of objects are passed to a function that see which Zmin-Zmax has greater value. The object on the top should be in the upward positive direction. 


\paragraph{next to}


\begin{figure}[H]
\centering
\includegraphics[scale=0.5]{images/Nextview.png}
\caption{}
\label{fig:next}
\end{figure}


A pair of objects next to each other have to have a distance not greater than 0.1 meters on the X and Y axis. The distance here is calculated for 2D corner points, this means the distance is calculated for four corners (min and max front and back).

The pairs should not have contained sides in neither the x nor the y axis. In this condition a pair of objects next to each other would like illustration A in \ref{fig:next}. This is might a bit different from what we consider next to each other as humans. We might imagine a typical next to pair as illustration B seen from view 1.

However, the choice of having 'next to' pairs not contained with each other is due to considerations of the view point. From view 1 the pair(B) seem next to each other but from view 2 they would not. In pair(B) from view 2, one object would be behind the other and likely hidden. So if view 1 is the global view and we pair the objects as in (B), seen as next to in view 1, the robot might enter the scene from view 2 and it would be wrong to refer to the pair as next to each other. However, if the 'next to' pairs are assigned as in illustration (A), the pair would be still visible in whichever view, and positioned proper enough to be referred to as next to each other. 

Finally the pair must have their lines contained on the Z axis. Otherwise, the two objects might satisfy the first condition on the (x,y) but be distant on the z axis, such as one object in the ceiling and the other on the floor. 


\paragraph{close to}

A pair close to each other are a pair who has any of their 3D corners close to each other within a max distance of 0.2 meter. No other conditions required for the pairing of objects close to each other. It can be a close object on, above, below or next to. 

\subsection{Second Module - question generation}


\begin{figure}[H]
\centering
\includegraphics[scale=0.5]{images/generator.png}
\caption{Split generator}
\label{fig:generator}
\end{figure}



 Our question generator generates questions of two types, size and spatial. In order to run the generator, the arguments required are the type of question, path to the val and train splits. 


 The question generator generates questions of one type at the time. Questions with the string "room" in is considered a different type from a question that refers to objects without a room. For example, the question "How big is the table?" has the type "Size", and the question "How big the table in the living-room?" is of type "Size room". In order to generate size questions with and without reference to a room, therefore, requires ruining the code, one separate time for each type. 

The split generator is the core component in the code. It takes a split either train/val and a question-type as arguments. The functionality of the split generator is to turn an EQA v1 split of episodes into a new split of new episodes. In figure \ref{fig:generator} we see an illustration of the workflow of the split generator. The five general steps is filtering(uncolored rotated square), iterator(red rectangle), episode parser (pink rectangle),  QA generator (the green rectangles), and  episode wrapper (the bottom yellow rectangle- inputs QA and outputs episode )  

The filter returns a set of question-episodes of one type only. The returned set of questions of a type is dependent on the question type given to generate. For example, if the input is to generate questions of 'size-room', 'how big is the sofa in the room', we take only the questions of "color-room" type. 

The filtered set is passed to an iterator. Each iteration passes one episode from EQA-v1 to a parser. The parser function in the iterator extract information, such as the object name and id, scene ID and room ID, from the EQA-V1 episode. 

The parsed information is passed to a QA generator. The QA generator is better described  as a group functions of the split-generator that are conditioned differently dependent on the question type. The answer generation function is, however, a different function for each question type.  

 The general idea for generating QA of any type relies on two straightforward steps. First, generating a ground-truth answer for the given question, which is the most important stage in the generation process as it requires calculating values from the data in the houses. Second step generating a question string and token ID's.

The  final step consists of inserting the new question with the corresponding geometric information, and structuring them into an executable function.  We call a QA sample an episode when the section of the episode seen in figure \ref{fig:episode} are filled with the new QA and the other the corresponding information. 

Our question generation can be described as generating one question for every “shortest path” there is. The idea of transforming  an episode from EQA into a new one is based on using the starting position and the 'shortest path' found in them. Having more questions for each shortest path is equivalent to having multiple questions about the same scene. 

\begin{figure}[H]
\includegraphics[scale=0.5]{images/stages.png}
\caption{Split generator (The top part of the code)}
\label{fig:stages}
\end{figure}


The top most part of the code (the data-set generator), illustrated in figure\ref{fig:stages} , passes the train and val to the split generator at different time-stamps. The reason for generating the two splits in two different stages is to keep track of the number of question-answers generated for each split. Emerging the two splits and splitting them randomly at the end might create and imbalance between the  answers in each split. The current code controls the distribution of answers in the train and val sets. Otherwise, leaving the type of answers uncontrolled would leave a bias towards one answer over the other. 

Once a split of episodes is generated it's passed to loader function. The loader function inserts the answer and question vocab to finalize the data-set in the form seen in figure 2.2

 In the coming two sub-section we describe in detail how the QA generators for the size and spatial questions work. 

\subsubsection{size-questions}

Size questions are generated through three steps. The first step is generating a ground truth answer about the size of the the target-object found in EQA-V1 episode. There three possible answers are  Big, Small, and Medium. The second step is generating a question string and token ids. The final step consist of filling the question-answer in an episode form, with shortest path and the rest of object's info from the original EQA-v1 episode, as described in the previous section. 

\subsubsection{Size answer} 

The size answer is generated in a function referred to as "GetsizeAnswer". This function takes as an argument the target-object's name and the size of its box and returns an answer about its size. The function calculates the volume of the target object in a similar way as the rest of the sizes of the objects. Volume of the OOBB = W x L x H. The next step in the function is to compare the size of the target to the sizes of the objects of its type.

The relative size is determined by its deviation from the standard of its type. As mentioned earlier the sizes of all objects are stored by type in a file. We pick the volumes of the object's type and calculate the mean size and the standard deviation of all the the sizes from the mean. The standard deviation denoted below: 

\[ s = \sqrt{\frac{1}{N-1} \sum_{i=1}^N (x_i - \overline{x})^2} 
\]

The answer is 'small' if the objects' size is smaller the mean size of its type minus the standard deviation,  'big' if the size is larger the median + the standard deviation, and middle if the size of the object is within the standard deviation added and subtracted from median. 

We control the answers' distribution. We observe that a majority of objects have a medium size given the standard of their type. In order to avoid bias towards the 'medium' answer, we restrict the number of QA with medium answer. We keep track of how many QA with medium answers has been generated and when the number reaches a limit we generate None QA that are later filtered out. The limit varies depending on the question type and the split (train or val), and is based on our observation of the answer distribution in the splits. Note that we refer to 'question type' in this example if either the question to be generated is size question with string 'room' such as "color-room" or without. 


\paragraph{question-string}

The question string generator takes a question type, and an object as an argument and returns a complete question string. 

The templates for size questions are as the following: 

size\_obj : 'how big <AUX> the <OBJ> ? \\
size\_room:'how big <AUX>  the <OBJ> in the <ROOM>?'

\subsubsection{spatial-questions}

Generating spatial question takes more complex steps and longer time than generating size questions. Generating a spatial QA requires a coordination with the spatial relation extractor. In addition, spatial questions include the addition of an extra object to the question string, and the insertion of the new object's information into the QA episode. 

Searching for spatial relations of the target object in an EQA episode is the first step taken. We pass the scene an room id to the 'relation' extractor to obtain pairs of objects, within a room,  with a spatial relation between them. The relation extractor returns three types of relational : next, on, or close, if existent within a room. Else it returns a category with empty values. 

The decision of generating a question of one of the mentioned relational categories is dependent on the existent of an object with a spatial relation to the target object. The process of executing a generation command of a question of a spatial type is illustrated in figure \ref{fig:spatial}. If there is an object 'on' the target or a target is on another object, we generate one questions, and similar case if there is an object next to the target object. If there is no 'on' or 'next' relation or either of them is non existent, the criteria for checking if there is a 'close' object is satisfied. If none of the conditions are satisfied a QA with no 'answer' of a random spatial type is generated.   

\begin{figure}[h]
\includegraphics[scale=0.4]{images/spatialconditions.png} %[scale=0.2]
\caption{Decision tree for generating different types of spatial questions}
\label{fig:spatial}
\end{figure}

A QA with positive spatial answer has a 'yes' answer, and 'no' if a relation is non existent. The decision tree as seen \label{fig:spatial} leverages positive QA for the reason that we observe that the no-relation instances outnumber the positive ones. The final condition, we even control the number of QA with 'no' answer by generating a None QA if the number of generated QA with no answer reaches a limit. The QAs' with None values are later filtered out.

Within this decision structure, for each 'shortest path' in an EQA episode, there is a possibility for generating from one to two spatial questions of different spatial type. 

The process of generating a spatial question includes the addition of information about two objects. An episode/question generator, a group of functions, adjust itself to a spatial question generation if certain arguments are given to it. These arguments are seen in the input section  in the illustration in fig \ref{fig:spatialGen}. Such as potential  object type, spatial question type, and all object in a room 

\begin{figure}[h]
\includegraphics[scale=0.4]{images/spatialGenerator.png} %[scale=0.2]
\caption{Structure of spatial questions generator}
\label{fig:spatialGen}
\end{figure}

All the inputs seen in \ref{fig:spatialGen} are required to generate an answer from a function called "GetSpatialAnswer". All objects in a room are needed for generating "no" answer. In case of generating a "no" answer the "GetSpatialAnswer" function picks a random object fro the houses that is not in the room. The reason of excluding objects in the room from the selection of a random object for a negative QA is to help us in the validation process, such as we would know if the robot answer 'yes' to a QA with 'no' as ground truth that it's due to bias rather than he robot recognizing the object in the scene. 

A selected object of potential objects and the type of spatial question are required arguments for generating spatial question string and token ids. 

The last part in blue is conditioned by the type of answer, if it's 'yes' or 'no'. If the answer is yes, geometric information of the target object's pair is passed to it to insert it in the episode. If the answer is 'no', no additional information is added to the episode beside the information of the target object found at the end of the shortest path. 
\paragraph{Question strings}

To generate a spatial question string, the string generator takes as an argument the question type such as "spatial\_room" and and spatial relation type such as "on", "next" or "close". The string templates are the following: 

'<AUX> there <ARTICLE> <OBJ1> close to the <OBJ> in the <ROOM>?'/\\
'<AUX> there <ARTICLE>  <OBJ1> next to the <OBJ> in the <ROOM>?' \\
'<AUX>  there <ARTICLE> <OBJ1> on the <OBJ> in the <ROOM>?\\
'<AUX> there <ARTICLE> <OBJ1> close to the <OBJ>?',\\
'<AUX>  there <ARTICLE> <OBJ1> next to the <OBJ> ?',\\
'<AUX>  there <ARTICLE> <OBJ1> on the <OBJ>?'


\subsection{Results}

\subsubsection{Total number of generated questions}

\begin{figure}[H]
\includegraphics[scale=0.25]{images/GenTrain.png}
\includegraphics[scale=0.25]{images/GenVal.png}
\caption{}
\label{fig:questionGen}
\end{figure}

We generate a total of 13 409 question for train and 2 335 questions for validation.  In figure \ref{questionGen} questions of size\_room and spatial\_room refer to questions that contains a reference to a room, such as 'How big is the bed in the bedroom?'. Questions of spatial\_obj or size\_obj type are questions with a reference to object only, such as "Is there a chair next to the table?". 

\subsubsection{Answers distribution}

\paragraph{Answers distribution of spatial questions}

\begin{figure}[H]
\includegraphics[scale=0.45]{latex/images/TrAnSp.png}
\includegraphics[scale=0.45]{latex/images/VlAnSp.png}
\label{fig:AnsDist}
\caption{}
\end{figure}

The majority of answers for the spatial questions are positive "yes" as seen in figure \ref{fig:AnsDist}. This imbalanced distribution is an intended outcome. The motivation behind this intention is based on an idea, formulated by \cite{regier1996human}, of learning of positive samples only. The argument behind this approach to learning  is inspired from a cognitive theory of human's first acquisition of language. The theory is  based on the premise that humans tend to learn spatial relations from positive evidence instead of non existent instances.

\paragraph{Answers distribution of size questions}

\begin{figure}[H]
\includegraphics[scale=0.45]{latex/images/TrAnSi.png}
\includegraphics[scale=0.45]{latex/images/VlAnSi.png}
\caption{}
\label{fig:AnsDist}
\end{figure}

The outcomes of generating size questions resulted with zero samples of "small" answer and a majority of 'medium' answer as seen in figure \ref{fig:AnsDist}. In the QA generator, we intended to limit the question-answers with "Medium" answer based on an observation of their dominance. However, limiting the 'medium' answers more than the presented numbers would have resulted in a very few question-answers of size type. An insignificant proportion of size questions was insufficient for training the model . We decided to keep the size questions with their imbalanced distribution, despite knowing that this linguistic bias might hinder the learning outcomes.  

\subsubsection{Discussion}

Our question-answer generation is very dependent on the objects and the shortest paths found in EQA-V1 data-set. The geometric information such as  starting positions and shortest paths are required material for extracting scenes training and testing the VQA model. To be only dependent on the objects found in the EQA restricts our choice over the QA that we could generate. The most prominent limitation we see in the generated questions is the inability to control the answer distribution for size questions. None of the target object's found in EQA V1 turned to be of a small size, based on the criteria we establish for determining an object's size. 

The description of sizes is paradoxical. A well established paradox, in philosophy


Topic sentence: we add spatial questions for the goal of .... deespite haing lingustic bias? 

Evaluation depending on no answers


Our choice to include size and spatial questions is motivated by the belief of the importance of spatial language to cognition. (\cite{landau1993whence}) in “spatial language and spatial cognition” states that the human first acquisition of linguistic names of objects in the physical world is associated with establishing a geometric representation of what defines them. In particular, the conceptual identification of an object might be defined within a spatial relation to other entities, and the image we mentally construct of a concrete noun of a physical property, may appear in the form of its shape.

The annswer 

Our question-answer generation is very dependent on the "shortest paths" in EQA-V1 data-set and the objects they lead to.  






\section{Task Two- Question Asking (Text to be added)}
\label{sec:task2}




\subsection{Training}

In the VQA model on the new questions with 50 epochs. In particular, train the LSTM with attention on the visual features, and use a pre-trained CNN for encoding visual features. The pre-trained CNN we use is one proposed by the researchers in habitat-platform and could be found on EQA gihub page.\footnote{ Link to the code for running the VQA baseline model. The same page include an attachment to the pre-trianed-CNN "\url{https://github.com/facebookresearch/habitat-lab/tree/master/habitat_baselines/il\#eqa-cnn-pretrain-model}.} 

We pick the model trained on the last epoch. The model has 1.10 average loss and  0.79 average accuracy. %, 2.21. % average mean rank, and 0.73 average reciprocal rank. 


\subsection{Evaluating}


The overall evaluation on the validation set with all the question types show an average accuracy of 0.61 and average loss 2.10. However, the accuracy score is by no means indicating a good system performance. As mentioned earlier, VQA system could cheat its ways by remembering answers and score high in accuracy results. These scores are attributed to a great extent to the bias we have in the size questions, where most of the answers to size questions are "medium". The absolute bias in the size question contribute to a higher score in the average accuracy of the overall validation. 


The results of the size questions showed all the predictions to be of 'medium' answer. In figure \ref{fig:heatmapSize}, the illustration of the predictions shows that all the answers to size questions been predicted as "medium". The results of the evaluation of the size questions are not surprising given the significant imbalance in their answer distribution. 


\begin{figure}[H]
\includegraphics[scale=0.3]{images/heatmapSize.png}
\caption{A map shows the number of predictions for each question type sorted by answer. All answers for size questions were predicted "medium". The row an column with big categories are zero}
\label{fig:heatmapSize}
\end{figure}



A reliable measure of evaluation for spatial questions is measuring the performance on the questions of "no" answers. The bias to 'yes' answers  in spatial questions was an intended learning experiment, and it is predictable that the system would preform well in predicting the questions with 'yes' answer. However, the extent in which the model's predictions are  based on exploiting bias could be measure by how many questions with "no" answer have been predicted correctly. Otherwise, given the bias of the answers to "yes", an indicator of learning shortcomings would be the number of times the model answered 'yes' to a question with 'no' answer. A positive learning indicator is the number of times that the system  predicted question with 'no' answer correctly. The system's prediction to questions with 'no' answer, therefore, provides a good validation measurement to how well the VQA model learnt about spatial relations. 



  

\begin{figure}[H]
 \centering
\subfloat[label 1]{{\includegraphics[width=5cm]{images/heatmapSpa.png} \label{<figure1>}}}
\subfloat[label 2]{{\includegraphics[width=5cm]{images/SpScores.png}\label{<figure2>} }}

\caption{A map shows the number of predictions for each question type sorted by answer.}
\end{figure}




In the section below we see the difference in the distribution of answer-predictions of the color questions between the original model and the model after being trained on the new questions. 

\begin{figure}[H]
\includegraphics[scale=0.45]{images/Heatmapbefore.png}
\label{fig:Heatmapbefore}
\caption{Predictions of color questions for the original model, untrained with new questions}
\end{figure}

\begin{figure}[H]
\includegraphics[scale=0.45]{images/HeatmapAfter.png}
\label{fig:HeatmapAfter}
\caption{Predictions of color questions on the model trained with new questions}
\end{figure}





\subsection{Discussion}

We should treat neural models as a theory or brain that is applicable and functional for every task scenario \cite{regier1996human}  

\cite{dobnik2009teaching}

Hcolor questions could get more complex as "people employ compositional color descriptions to express meanings not covered by basic terms, such as greenish-blue" \cite{monroe2016learning}. It would be shallow to assume that color questions are simplistic, especially if we expect the system to answer colors beyond the basic color terms like "green" and "red." 

\cite{monroe2017colors}


intersective compostionality: intersective compositionality is when two words which one is attributed to the meaning of the other "brown bear" where it means a bear that is brown-[brown and bear] 

non-intersective compostionality: non-intersective is one word does not modify the second, such as [Teddy bear]. 'Teddy bear' cannot be mean a bear that is 'Teddy', 'Teddy' is not an attribute of a bear so not [Teddy + bear]. Teddy + bear is instead a different entity with a different perceptual meaning. \cite{larsson-2017-compositionality} 




\addcontentsline{toc}{section}{References}
\bibliography{references}

\newpage
\section{Appendices}


\subsection{datasets} 

The 3D Scenes and the QA dataset mentioned in \cite{embodiedqa}, are called SUNCG(3D houses) and "EQA V1" (QA). The EQA V1 is a synthetic dataset generated automatically, and constructed based on the setting of the 3D houses in SUNCG. SUNCG is no longer available. \cite{embodiedqa} changed the SUNCG 3D setting to MatterPort 3D (MP3D). MatterPort 3D is a reconstruction of 3D houses in (SUNCG) scene dataset. The latter also implies that the inital "EQA V1" is not applicable for MP3D. 

The new QA dataset for Matterport 3D is available but not the code that generated it. The EQA-mp3d v1 is also a synthetic dataset generated automatically and can be found at this footnote reference \footnote{https://github.com/facebookresearch/habitat-lab}. For generating  questions for SUNCG, a code published at this reference\footnote{https://github.com/facebookresearch/EmbodiedQA}. However, there is no code for generating QA for MP3D. 
 
A few of the differences between the question dataset for SUNCG (EQA-SUNCG) and MP3D(EQA-MP3d) are mentioned in \cite{eqa_matterport}. However, not in all the information in  \cite{eqa_matterport} seems to match with EQA-MP3D that we have. In  \cite{eqa_matterport} page(4) it's stated that the number of scene used from MP3D is 76. The dataset we downloaded from "facebookai/habitat" repo on github uses a total 67 scene of 90 scenes available in MatterPort3D. 57 of the 67 scenes are used for questions in the train-set and 10 in the the enviroment. Note that the latter implies that the robot is tested on different scenes from the scenes it has been trained in. 


\subsubsection{List of textual references with number of answer choices}

. 
\begin{figure}[H]
    \centering
    \subfloat[]{{\includegraphics[width=6cm]{images/plot1.png}}}%
    %\qquad
    \subfloat[]{{\includegraphics[width=6cm] {images/plot2.png}}}%
    \caption{Each row is a textual reference. The number of answer choices for each textual reference is represented by the colorful blocks. One block = 1 answer choice and so on. The colors of the bars are not representative of the named color answer so they should be disragarded}
\end{figure}.



\subsection{Habitat-lab(EQA evaluation)}

\begin{figure}[H]
\centering
\includegraphics[scale=0.43]{images/configProcess.png}
\caption{Example of Habitat lab processing the configurations to implement validation for the VQA model }
\label{fig:configs}
\end{figure}

Figure \ref{fig:configs} resembles a map of the code structure when the habitat lab module is initiated to preform validation task for the VQA. Each task has its own configurations and in this example the task is 'VQA evaluation'. As seen in the figure \ref{fig:configs}, the module takes hierarchical steps in which each step is executed in accordance to the configuration of the given task. In the most down box of the structure we see parts of the commands directed for the simulator, such as insinuating an environment and sensors in the agent. Other commands include registering a data-set which takes part in lab module. 


The configurations are processed into commands in Habitat-lab before being passed to the simulator. Habitat lab is the second core component of the system. In addition to giving commands to the simulator, the Habitat Lab module acts as a pipeline that prepares the data-set of the corresponding task. The habitat-lab module,in other words, is the coordinator that informs the simulator of the required setting, and the data loader and processor that prepares the data for either training or testing. 



y in science and
engineering (from aerospace to zoology). In the context of
embodied AI, simulators help overcome the aforementioned
challenges – they can run orders of magnitude faster than
real-time and can be parallelized over a cluster; training
in simulation is safe, cheap, and enables fair comparison
and benchmarking of progress in a concerted communitywide effort. Once a promising approach has been developed
and tested in simulation, it can be transferred to physical
platforms that operate in the real world

. The name 'Habitat' is derived from  the notion of learning within and from an environment. Imitating our natural habitat, the Habitat platform facilitates spawning an agent in a simulated environments with the possibility of teaching the robot to preform different tasks. 

The perquisites needed to test or train an agent for a certain task in a given environment, are facilitated by a core component called  Habitat Simulator. Habitat Simulator is responsible for simulating an  environment and insinuating a robot in it. The simulator acts depending on the configurations given to it. 



\end{document}


